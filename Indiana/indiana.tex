\documentclass[11pt]{article}

%%%%%%%%%%%%%%%%%%%%%%%%%%%%%%%%%%-- Settings --%%%%%%%%%%%%%%%%%%%%%%%%%%%%%%%%%%%%%%%%%%%
\usepackage[english]{babel}

% - Margin - 1 inch on all sides
\usepackage[letterpaper]{geometry}
\usepackage[utf8]{inputenc}
\geometry{top=0.5in, bottom=0.5in, left=0.5in, right=0.5in}

% Suppress Page numbers
\pagenumbering{gobble}

% Other Packages
\usepackage{outlines}

\renewcommand{\footnoterule}{%
  \kern -3pt
  \hrule width \textwidth height 0.5pt
  \kern 2pt
}

%%%%%%%%%%%%%%%%%%%%%%%%%%%%%%%%%%%%-- Personal Statement --%%%%%%%%%%%%%%%%%%%%%%%%%%%%%%%%%%%%%%


%%%%%%%%%%%%%%%%%%%%%%%%%%%%%%%%%%%%%%%%-- Document --%%%%%%%%%%%%%%%%%%%%%%%%%%%%%%%%%%%%%%%%%%%%

% Begin Document
\begin{document}
\begin{center}
  \textbf{Mathematical Preparation and Taste}\\
  James Harbour \\
  Application to \emph{Indiana} REU
\end{center}

In mathematics, one often has a statement which should be true but evades any seemingly natural methods of proof. What excites me most in mathematics is when the correct method for attacking such statements is the use of surprising reductions to facts that come from completely different disciplines. Mainstay canon such as Riemann's mapping theorem and Burnside's $pq$-theorem are interesting statements on their own; however, the true meat of why I love these theorems is that their proofs both rely on an incredibly creative reduction to facts or properties from fields that seem, at least at first, entirely unrelated to the original statements.

My true mathematical journey started as a junior in high school when I began dual-enrolling in three courses per semester at the University of South Florida. After completing my first ``introduction to proofs'' course, I dipped my toes into analysis and algebra. I enjoyed both equally and wished to experience them at the graduate-level, so I went about convincing my county's board of education to allow me to take graduate courses for dual-enrollment. As the graduate-level analysis courses were offered in the mornings, a time during which I was required by law to attend high school every day, I was pushed towards entering the University of South Florida's graduate algebra sequence.

It was during this sequence of courses that I first encountered one of those aforementioned facts: Burnside's $pq$-theorem. The result was mentioned in a passing comment whilst covering Sylow's theorems, and what stuck with me was that its most straightforward proof relied upon representation theory. The fact that such a simply stated theorem in group theory required reaching far beyond the machinery my algebra course had built ignited my interest. This connection led me to take a course on representation theory during my first semester at the University of Virginia at the end of which we covered the proof of Burnside's $pq$ theorem in full. One aspect of the proof that really exemplifies my aforementioned mathematical taste is the reduction of the original theorem statement to showing the existence of a nontrivial complex representation $\rho$ and nonidentity group element $g$ such that $\rho(g)$ is a scalar operator. I was immediately attracted to this big-picture idea of taking something purely group-theoretic and turning it on its head by reaching outside of pure group theory to obtain an entirely non-obvious connection to a field whose tools make the problem feel much more approachable.

In addition to representation theory, during my first semester at the University of Virginia the mathematics department gave me the freedom to push myself right out of the gate taking almost an identical course load to the university's first-year graduate students. My major takeaways from these courses are twofold. First, through my graduate complex analysis course I grew accustomed to an ``in the weeds'' style of doing analysis. Professor Benjamin Hayes preferred the use analytical tools of complexity approximately equal to that of the theory at hand and encouraged our adoption of this philosophy. For example, in one homework we were tasked with justifying the holomorphicity of the limit of a locally uniformly convergent sequence of holomorphic functions; however, we were not allowed to use Morera's theorem for a quick proof and had to proceed using more fundamental results such as the Cauchy integral formula. This course also solidified my mathematical taste through our development of normal families for proving Riemann's mapping theorem. Professor Hayes mentioned that this statement was one of the main goals of the course; however, for a large portion of the course I could not see any way forward with the tools we had developed. What I loved about normal families is that they provided a concrete framework for proving statements such as the Riemann mapping theorem by borrowing from functional analysis, an entirely different field, in an ingenious manner.

Second and most importantly, I experienced just how invigorating collaborating (and struggling) on tough problems with a cohort of people who are just as determined and passionate about math as I am is. From the outset of the semester, I began to work on mathematics together with my new classmates, some of whom were advanced undergraduates and the rest first-year graduate students. Gradually, I became part of a tight-knit group of fellow students who, regardless of age or background, were all working through the same concepts I was and shared the same passion I had. Whether it be in the form of desperately brainstorming for ideas to tackle a problem or intently listening to each other discuss fascinating research over pizza, this past semester I have learned that there is nothing more rewarding than doing mathematics with others.

\vspace{0.25cm}
\noindent Regards,

\vspace{0.25cm}
\noindent James Harbour








\end{document}
