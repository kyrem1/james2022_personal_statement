
\documentclass[11pt]{article}

%%%%%%%%%%%%%%%%%%%%%%%%%%%%%%%%%%-- Settings --%%%%%%%%%%%%%%%%%%%%%%%%%%%%%%%%%%%%%%%%%%%
\usepackage[english]{babel}

% - Margin - 1 inch on all sides
\usepackage[letterpaper]{geometry}
\usepackage[utf8]{inputenc}
\geometry{top=0.5in, bottom=0.5in, left=0.5in, right=0.5in}

% Suppress Page numbers
\pagenumbering{gobble}

% Other Packages
\usepackage{outlines}

\renewcommand{\footnoterule}{%
  \kern -3pt
  \hrule width \textwidth height 0.5pt
  \kern 2pt
}

%%%%%%%%%%%%%%%%%%%%%%%%%%%%%%%%%%%%-- Personal Statement --%%%%%%%%%%%%%%%%%%%%%%%%%%%%%%%%%%%%%%


%%%%%%%%%%%%%%%%%%%%%%%%%%%%%%%%%%%%%%%%-- Document --%%%%%%%%%%%%%%%%%%%%%%%%%%%%%%%%%%%%%%%%%%%%

% Begin Document
\begin{document}
\begin{center}
  \textbf{Relevant Programming Experience}\\
  James Harbour \\
  Application to \emph{Indiana} REU
\end{center}

I am competent with both SageMath and Python, as well as Mathematica and Java. During the my modern geometry course, I used SageMath and Mathematica combined with Pontryagin's maximum principle to study the geodesics of the Heisenberg group, the Grushin plane, and the Engel group as sub-Riemannian manifolds, so I have experience with solving genuine problems utilizing these computer algebra systems. For formal training, I have taken multiple computer science courses utilizing Java and python. In regards to self-motivated programming projects, I have implemented some basic reinforcement-learning algorithms in python. Additionally, I followed Kenneth Stanley's classic paper \emph{Evolving Neural Networks through Augmenting Topologies} to implement my own rendition of his NEAT algorithm in Java.\\

\noindent Regards,

\vspace{0.25cm}
\noindent James Harbour



\end{document}
