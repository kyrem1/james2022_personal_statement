\documentclass[12pt]{article}

%%%%%%%%%%%%%%%%%%%%%%%%%%%%%%%%%%-- Settings --%%%%%%%%%%%%%%%%%%%%%%%%%%%%%%%%%%%%%%%%%%%
\usepackage[english]{babel}

% - Margin - 1 inch on all sides
\usepackage[letterpaper]{geometry}
\usepackage[utf8]{inputenc}
\geometry{top=1.0in, bottom=1.0in, left=1.0in, right=1.0in}

% Indent First paragraph

% Other Packages
\usepackage{outlines}

\renewcommand{\footnoterule}{%
  \kern -3pt
  \hrule width \textwidth height 0.5pt
  \kern 2pt
}

%%%%%%%%%%%%%%%%%%%%%%%%%%%%%%%%%%%%-- Personal Statement --%%%%%%%%%%%%%%%%%%%%%%%%%%%%%%%%%%%%%%


%%%%%%%%%%%%%%%%%%%%%%%%%%%%%%%%%%%%%%%%-- Document --%%%%%%%%%%%%%%%%%%%%%%%%%%%%%%%%%%%%%%%%%%%%

% Begin Document
\begin{document}
\begin{center}
  \textbf{Personal Statement}\\
  James Harbour \\
  Application to \emph{insert name here} REU
\end{center}

% Try to build around this spider web analogy

Regardless of one's feelings towards spiders, one must admit that elaborate spider webs possess a certain beauty to them. In viewing mathematics as a colossal spider web, what excites me is discovering those hidden threads which are in fact as strong as steel. I crave those incredible coincidences which are actually anything but coincidences. Why do conditions that imply a family of analytic functions from a fixed region is normal always seem to imply that a given entire function is constant? Why are nonabelian groups of order a product of two prime powers never simple?



% Maybe take algebraic geometry and graph theory as examples of this spider web that are polar opposites as subjects ::: or ditch this idea


% from Complex analysis: Normal Families and no nonconstant entire functions ---> Bloch's principle.
% from Group Theory: Burnside's pq theorem, Schur's lemma, and group algebras.
% regular n-gon is constuctible if and only if the each of the prime factors of n are either 2 or a Fermat prime.



My true mathematical journey started as a junior in high school when I began dual-enrolling in three courses per semester at my local university (the University of South Florida). After completing my first ``introuduction to proofs'' course, I dipped my toes into analysis and algebra. I enjoyed both equally and wished to experience them at the graduate-level, so I went about convincing my county's board of education to allow me to take graduate courses for dual-enrollment. As the graduate-level analysis courses were offered in the mornings, a time during which I was required by law to attend high school every day, I was pushed towards entering the University of South Florida's graduate algebra sequence.

It was during this sequence of courses that I first encountered an assortment of those aforementioned hidden threads. We covered some of these connections, such as the fact that a regular $n$-gon is constuctible if and only if each of the distinct prime factors of $n$ are either $2$ or a Fermat prime; however, other facts like Burnside's $pq$-theorem were left unaddressed and only mentioned in passing. These threads stuck to my mind like a ..., resurfacing in my thoughts to torment me from time to time and leading me to seek out explanations for myself. Burnside's $pq$-theorem led me to take a course on representation theory. The link between normal families and the nonexistence of nonconstant entire functions with certain prescribed propertains led me to begin reading Miranda's book on Riemann surfaces. In seeking and understanding these connections for myself, my mathematical taste developed into what it is today.

% TODO transition

[TODO insert smart transition here] For my first semester at the University of Virginia, the mathematics department gave me the freedom to push myself right out of the gate taking almost an identical courseload to the university's first-year graduate students. My major takeaways from this semester are threefold. First, from my graduate algebra course I gained an appreciation for the numerous benefits of maintaining a categorical perspective.
Whilst teaching this course, Professor Andrei Rapinchuk constantly emphasized the categorical/functorial properties of our objects of study and demonstrated the usefulness of category theory as a bookkeeping technique. Second, through my graduate complex analysis course I grew accustommed to an ``in the weeds'' style of doing analysis. [TODO explain]

% TODO explain Analysis in the weeds

% TODO Talk about experience of doing math with other people.

Third and most importantly, I experienced just how invigorating collaborating (and struggling) on tough problems with a cohort of people who are just as determined and passionate about math as I am is. From the outset of the semester, I began to work on mathematics together with my new classmates, some of whom were advanced undergraduates and the rest first-year graduate students. Gradually, I became part of a tight-knit group of fellow students who, regardless of age or background, were all working through the same concepts I was and shared the same passion I had. Whether it be [insert something here about people getting crazy ideas to solve problems] or intently listening to each others discuss fascinating research over pizza, this past semester I have learned that there is nothing more rewarding than doing mathematics with others.
\\
% Expand on the above
% experience working with classmates, "some around my age, some 10 years older than me, it made no difference to the feeling of belonging [find better word, something like collectiveness or something]"
% I became part of a tight-knit group of fellow students,

[TODO maybe insert transition phrase/sentence] This upcoming semester, I will continue to push myself by again taking the standard courses for UVA's first-year graduate students: algebraic topology, measure theory, and algebra II.
\\

% TODO talk about plan to take the analysis, topology, and algebra general exams

% TODO maybe talk about how I enjoy playing around with things.

% TODO maybe talk about self study Tao Measure theory, Evans PDEs


% TODO talk about future goals/plan.



\end{document}
