\documentclass[11pt]{article}

%%%%%%%%%%%%%%%%%%%%%%%%%%%%%%%%%%-- Settings --%%%%%%%%%%%%%%%%%%%%%%%%%%%%%%%%%%%%%%%%%%%
\usepackage[english]{babel}

% - Margin - 1 inch on all sides
\usepackage[letterpaper]{geometry}
\usepackage[utf8]{inputenc}
\geometry{top=0.8in, bottom=0.8in, left=1.0in, right=1.0in}

% Suppress Page numbers
\pagenumbering{gobble}

% Other Packages
\usepackage{outlines}

\renewcommand{\footnoterule}{%
  \kern -3pt
  \hrule width \textwidth height 0.5pt
  \kern 2pt
}

%%%%%%%%%%%%%%%%%%%%%%%%%%%%%%%%%%%%-- Personal Statement --%%%%%%%%%%%%%%%%%%%%%%%%%%%%%%%%%%%%%%


%%%%%%%%%%%%%%%%%%%%%%%%%%%%%%%%%%%%%%%%-- Document --%%%%%%%%%%%%%%%%%%%%%%%%%%%%%%%%%%%%%%%%%%%%

% Begin Document
\begin{document}
\begin{center}
  \textbf{Statement of Purpose}\\
  James Harbour \\
  Application to \emph{UC Davis} REU
\end{center}





\noindent\textbf{(a): Mathematical Taste and Background}\\

In mathematics, one often has a statement which \emph{should} be true but evades any seemingly natural methods of proof. What excites me most in mathematics is when the correct method for attacking such statements is the use of surprising reductions to facts that come from completely different disciplines. Mainstay canon such as Riemann's mapping theorem and Burnside's $pq$-theorem are interesting statements on their own; however, the true meat of why I love these theorems is that their proofs both rely on an incredibly creative reduction to facts or properties from fields that seem, at least at first, entirely unrelated to the original statements.

My true mathematical journey started as a junior in high school when I began dual-enrolling in three courses per semester at my local university (the University of South Florida). After completing my first ``introduction to proofs'' course, I dipped my toes into analysis and algebra. I enjoyed both equally and wished to experience them at the graduate-level, so I went about convincing my county's board of education to allow me to take graduate courses for dual-enrollment. As the graduate-level analysis courses were offered in the mornings, a time during which I was required by law to attend high school every day, I was pushed towards entering the University of South Florida's graduate algebra sequence.

It was during this sequence of courses that I first encountered one of those aforementioned facts: Burnside's $pq$-theorem. The result was mentioned in a passing comment whilst covering Sylow's theorems, and what stuck with me was that its most straightforward proof relied upon representation theory. The fact that such a simply stated theorem in group theory required reaching far beyond the machinery my algebra course had built ignited my interest.

This connection led me to take a course on representation theory during my first semester at the University of Virginia at the end of which we covered the proof of Burnside's $pq$ theorem in full. One aspect of the proof that really exemplifies my aforementioned mathematical taste is the reduction of the original theorem statement to showing the existence of a nontrivial complex representation $\rho$ and nonidentity group element $g$ such that $\rho(g)$ is a scalar operator. I was immediately attracted to this big-picture idea of taking something purely group-theoretic and turning it on its head by reaching outside of pure group theory to obtain an entirely non-obvious connection to a field whose tools make the problem feel much more approachable.


In addition to representation theory, during my first semester at the University of Virginia the mathematics department gave me the freedom to push myself right out of the gate taking almost an identical course load to the university's first-year graduate students. My major takeaways from these courses are threefold. First, from my graduate algebra course I gained an appreciation for the numerous benefits of maintaining a categorical perspective. Whilst teaching this course, Professor Andrei Rapinchuk constantly emphasized the categorical/functorial properties of our objects of study and demonstrated the usefulness of category theory as a bookkeeping technique.

Second, through my graduate complex analysis course I grew accustomed to an ``in the weeds'' style of doing analysis. Professor Benjamin Hayes preferred the use analytical tools of complexity approximately equal to that of the theory at hand and encouraged our adoption of this philosophy. For example, in one homework we were tasked with justifying the holomorphicity of the limit of a locally uniformly convergent sequence of holomorphic functions; however, we were not allowed to use Morera's theorem for a quick proof and had to proceed using more fundamental results such as the Cauchy integral formula. This course also solidified my mathematical taste through our development of normal families for proving Riemann's mapping theorem. Professor Hayes mentioned that this statement was one of the main goals of the course; however, for a large portion of the course I could not see any way forward with the tools we had developed. What I loved about normal families is that they provided a concrete framework for proving statements such as the Riemann mapping theorem by borrowing from functional analysis, an entirely different field, in an ingenious manner.

Third and most importantly, I experienced just how invigorating collaborating (and struggling) on tough problems with a cohort of people who are just as determined and passionate about math as I am is. From the outset of the semester, I began to work on mathematics together with my new classmates, some of whom were advanced undergraduates and the rest first-year graduate students. Gradually, I became part of a tight-knit group of fellow students who, regardless of age or background, were all working through the same concepts I was and shared the same passion I had. Whether it be in the form of desperately brainstorming for ideas to tackle a problem or intently listening to each other discuss fascinating research over pizza, this past semester I have learned that there is nothing more rewarding than doing mathematics with others. \\

% The  would provide an ideal extension of this experience to working on topics at the frontiers of research. Moreover, due to its high caliber, the experience would foster my mathematical growth immensely. My interest towards Knot Theory project began during the previous semester when I attended an incredibly fascinating talk on knot Floer homology and was excited by the subject's use of encoding purely topological data using clever combinatorial manipulations and algorithms. I would be delighted to pursue research in the subject at this program.

% In the long term, the UC Davis REU would provide a strong foundation for me to eventually begin my own independent mathematics research as an undergraduate and to be able to focus heavily on research right from the start of graduate school. Thank you for your consideration. \\

% \noindent Regards, \\
%
% \noindent James Harbour
\noindent\textbf{(b): Interest in the Program}\\

What I love about graph theory is it practically forces you to do mathematics in a way that I find appealing; the core definitions are so barebones that you have to constantly reach outside of the subject itself for techniques. The first graduate-level textbook I ever took the time to truly digest over long periods of time was Reinhard Diestel's \emph{Graph Theory}. Every topic I would learn during the course of my mathematical education would have some type of application to graph theory; whether it be linear algebra, abstract algebra, or probability theory. I would love to use what I learned throughout this endeavor to delve into the deterministic structures arising from sum-to-product games.

I was also led towards learning about error-correcting codes during a foray in to the sporadic finite simple groups. After seeing the automorphism groups of various codes and designs appear in the constructions of the Mathieu groups, I dove headfirst into MacWilliams and Sloane's \emph{The Theory of Error-Correcting Codes}. Through this text, I learned the theory behind various combinatorial objects such as Steiner systems (and designs in general) as well as the Hamming and Golay codes. I would love to take what I have learned through this exploration and apply it to the theory of quantum error correction in UC Davis's project on quantum error correction in novel quantum geometries.\\


\noindent\textbf{(c): Statement of Diversity and Inclusion}\\


Mathematics is the crystallization of pure logical thought and progress. I believe that discriminatorily rejecting other’s opinions or participation based upon any categorical standard is entirely contrary to the spirit of mathematics as a discipline. Only through collaboration with people from all walks of life can we make progress.

During high school, I served as the coach for my school math team’s calculus team. I organized and gave weekly lectures on various topics not covered in the standard calculus curriculum in preparation for tournaments. One aspect of the lectures that I introduced to foster an inclusive environment was a post-lecture problem solving session. I would use a random number generator to select a group of people to present their solutions to the previous week’s assigned problems. To relieve stress, I allowed people to opt out of this procedure; however, the team loved it and grew closer as they were able to see the problem-solving strategies of classmates that would normally have not chosen to come to the board due to feeling a lack of inclusivity.




\end{document}
